\begin{slide}{Program and Cast}
\begin{description}
\item[ACT I ({\em Warming Up}):] \hspace{1in} \\
\begin{itemize}
\item Introduction and Soot Basics {\blue (Laurie)}
\item Intraprocedural Analysis in Soot {\blue (Patrick)}
\end{itemize}
\item[ACT II ({\em The Home Stretch}):] \hspace{1in} \\
\begin{itemize}
\item Interprocedural Analyses and Call Graphs {\blue (Ond\v{r}ej)}
\item Attributes in Soot and Eclipse {\blue (Ond\v{r}ej,Feng,Jennifer)}
\item {\red Conclusion, Further Reading \& Homework {\blue (Laurie)}}
\end{itemize}
\end{description}
\end{slide}

\begin{slide}{Conclusion}
\begin{itemize}
\item Have introduced Soot 
\end{itemize}
\end{slide}

\begin{slide}{Homework}
\begin{itemize}
\item Try out Soot
\begin{description}
\item[Super easy:]  Soot as a stand-alone tool,  Eclipse plugin
\item [Easy:]  implement a new intraprocedural analysis and
                 generate tags for it.  
\item [More challenging:]  implement whole program analysis,  toolkit or a
new IR.
\end{description}

\item Please stay in touch, tell us how you are using Soot and
contribute back any new additions you make.
\end{itemize}
\end{slide}

\begin{slide}{Resources and Further Reading}
\begin{description}
\item [Soot options:]
\item [Introduction to Soot (1.x):]  
Raja's thesis, CASCON 99, CC 2000, SAS 2000
\item [Small tutorials on Soot (still being updated for Soot 2.0):]
\item [Javadoc:] in main distribution
\item [Initial design of attributes:] CC 2001
\item [Array bounds checking elimination:] Feng's thesis, CC 2002
\item [Decompiling:] Jerome's thesis, CC 2003
\item [Points-to analysis:] Ondrej's thesis, CC 2003, PLDI 2003 (BDD-based)
\end{description}
\end{slide}
